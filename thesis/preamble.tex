% SNU Ph.D. Dissertation Template (English) — Updated to official 규격 (4×6배판)

% Source by 조형곤
% Revised by 박경수, 조형곤, 박철홍

% ---------------------------------------------------------------------------- %
%                                   PACKAGES                                   %
% ---------------------------------------------------------------------------- %


% ======= 페이지 레이아웃 =======
\usepackage[b5paper,
            top=2cm, bottom=1.5cm,
            left=3cm, right=3cm,
            headheight=15pt, headsep=15pt, footskip=15pt]{geometry}

% ======= 인코딩 및 글꼴 설정 =======
\usepackage[utf8]{inputenc}            % 입력 인코딩
\usepackage[T1]{fontenc}               % 출력 인코딩 (글꼴 인코딩)
\usepackage{kotex}                     % 한글 지원
\usepackage{newtxtext}                 % 영문 기본 글꼴 (Times New Roman (for English))

% ======= 참고문헌 =======
% \usepackage[round]{natbib}             % 참고문헌 인용 형식 (저자, 연도) (기존은 [숫자]임)
\usepackage[
  backend=biber,
  style=apa,   % apa   authoryear   ext-authoryear
  citestyle=apa,
  sorting=nyt,
  maxcitenames=1,   % 인용에서 보여줄 최대 저자 수
  mincitenames=1,   % 최소 두 명은 항상 이름 표시
  maxbibnames=99,   % 참고문헌 목록에 표시할 저자 수 (가능한 많이 유지)
  uniquename=false,
]{biblatex}

\addbibresource{reference.bib}

% 참고문헌 번호를 [1] 형식으로 보이게 조정
\DeclareFieldFormat{labelnumberwidth}{[#1]}
\renewcommand*{\labelnamepunct}{\addcolon\space}   % 강제로 번호 달기
\defbibenvironment{bibliography}
  {\begin{enumerate}
      \setlength{\labelwidth}{1.5em}
      \setlength{\leftmargin}{\labelwidth}
      \setlength{\labelsep}{0.5em}
      \setlength{\itemsep}{\bibitemsep}
      \setlength{\parsep}{\bibparsep}
   }
  {\end{enumerate}}
  {\item}


% ======= 시각 요소: 그림, 색상 =======
\usepackage{graphicx}                  % 그림 포함
\usepackage{xcolor}                    % 색상 지원
\usepackage[labelfont=bf, labelsep=period]{caption}   % : 대신 . 사용
\graphicspath{{./figures/}}            % 그림 경로 설정
\usepackage{subcaption}                % n by n 형식의 그림 배열
\usepackage{placeins}

% ======= 문단, 줄간격 등 레이아웃 설정 =======
\usepackage{setspace}                  % 줄간격 조정
\usepackage{indentfirst}               % 첫 문단 들여쓰기. 만약 개별문단별로 적용하려면 \indent 사용
\setstretch{1.7}                       % 기본 줄간격 1.7배 (170%)
\setlength{\parskip}{0.6em}            % 단락 사이 (세로) 여백
\setlength{\parindent}{2em}            % 단락 들여쓰기

% ======= 제목/번호/목차 관련 =======
\usepackage{titlesec}                  % 제목 포맷 커스텀
\usepackage{fancyhdr}                  % 헤더/푸터
\usepackage{tocloft}                   % 목차 스타일

% ======= 수식 및 수학 패키지 =======
% \usepackage{amsmath}                   % 수학 수식
\usepackage{amsmath, amssymb, amsfonts}
\usepackage{mathtools}  % for \coloneqq
% \numberwithin{equation}{chapter}       % 챕터별 수식 번호 (1.1)
% \renewcommand{\theequation}{Equation \thechapter.\arabic{equation}}   % (Equation 1.1)

% --- 표 관련 패키지 ---
\usepackage{booktabs}                  % \toprule 등 고급 표
\usepackage{longtable}                 % 페이지 넘기는 표
\usepackage{tabularx}                  % 가변 폭 표
\usepackage{multirow}                  % 셀 병합
\usepackage{makecell}                  % 셀 내 줄바꿈
\usepackage[tableposition=top]{caption}% 표 캡션 위 고정
\newcolumntype{C}{>{\centering\arraybackslash}X}

% ======= 참조 패키지 =======
\usepackage[hidelinks]{hyperref}       % 하이퍼링크
\usepackage[capitalise]{cleveref}      % cref 자동 참조
\usepackage{footmisc}                  % 각주 설정

% ======= 기타 디자인 =======
\usepackage{tcolorbox}                 % 박스 디자인

% ======= 코드류 =======
\usepackage{listings}                  % 코드삽입
\usepackage{lipsum}                    % 더미문장 \lipsum[1-2]


% ---------------------------------------------------------------------------- %
%                                    STYLE                                     %
% ---------------------------------------------------------------------------- %

% ======= Title formatting =======
\titleformat{\chapter}[hang]{\bfseries\LARGE}{Chapter\ \thechapter.}{1em}{}     % 장 제목
\titleformat{\section}[hang]{\bfseries\Large}{\thesection}{1em}{}               % 절 제목
\newcommand{\sectionbreak}{\clearpage}
\titleformat{\subsection}[hang]{\bfseries\large}{\thesubsection}{1em}{}    % 절의 하위제목
\titleformat{\subsubsection}[hang]{\bfseries\normalsize}{\thesubsection}{1em}{} % 하위제목의 하위제목
\titleformat{\paragraph}[hang]{\bfseries\normalsize}{\thesubsection}{1em}{}     % 소단락 제목
\titleformat{\subparagraph}[runin]{\bfseries\normalsize}{\thesubsection}{1em}{}  % 소단락의 하위제목

% ======= 서울대 맞춤 폰트 스타일 (필요시) =======
% \newcommand{\이름}[인자 수]{내용}
% \newcommand{\UnnumberedChapter}[1]{\fontsize{11pt}{18.7pt}\selectfont\textbf{#1}\\}   % {글꼴크기}{줄간격} 마지막 \\는 Enter 역할
\newcommand{\CustomToC}[1]{%
  \begingroup
    \fontsize{12pt}{20.4pt}\selectfont
    \setstretch{1.7}
    #1
  \endgroup
}   % 목차(Table of Contents) 스타일
\newlength{\basegap}        % 새 길이 변수를 선언 (표지에서 줄띄어쓰기용)
\setlength{\basegap}{14pt}  % 해당 변수에 14pt를 지정

% ======= Roman page numbers for frontmatter =======
\pagenumbering{roman}

% ======= Header/Footer =======
\pagestyle{fancy}
\fancyhf{}
\renewcommand{\headrulewidth}{0pt}
\fancyfoot[C]{\thepage}  % Footer centered


% ---------------------------------------------------------------------------- %
%                           TABLE AND FIGURE TEMPLATE                          %
% ---------------------------------------------------------------------------- %

\newcommand{\TableUppersideBlank}{10pt}
\newcommand{\TableLowersideBlank}{10pt}
\newcommand{\TableArrayStretch}{1.5}
\setlength{\belowcaptionskip}{\TableUppersideBlank}
\setlength{\textfloatsep}{\TableLowersideBlank}
\setlength{\LTpre}{\TableUppersideBlank}      % 테이블 위 (본문과의) 간격
\setlength{\LTpost}{\TableLowersideBlank}     % 테이블 아래 (본문과의) 간격

% --- 함수 1: 내용물 기준 테이블 (tabular) ---
% 사용법: \newcontenttable[<placement>]{<colspec>}{<caption>}{<label>}{<header-macro>}{<body-macro>}
% 예시:   \newcontenttable[ht!]{lcr}{캡션}{tab:my}{\myhead}{\mybody}
% ----------------------------------------------------------------------------
\newcommand{\newcontenttable}[6][htbp!]{%
  \begin{table}[#1]
    \centering
    \caption{#3}
    \label{#4}
    {%
    \renewcommand{\arraystretch}{\TableArrayStretch}
    \begin{tabular}{#2}
      \toprule
      #5 \\ % 헤더 매크로 확장
      \midrule
      #6   % 바디 매크로 확장 (마지막 \\는 바디 매크로 안에 포함되어야 함)
      \bottomrule
    \end{tabular}
    }
  \end{table}
}

% --- 함수 2: 좌우 꽉 찬 테이블 (tabularx) ---
% 사용법: \newfullwidthtable[<placement>]{<colspec>}{<caption>}{<label>}{<header-macro>}{<body-macro>}
% 예시:   \newfullwidthtable[ht!]{lX}{캡션}{tab:my}{\myhead}{\mybody}
% ----------------------------------------------------------------------------
\newcommand{\newfullwidthtable}[6][htbp!]{%
  \begin{table}[#1]
    \centering
    \caption{#3}
    \label{#4}
    {% 
      \renewcommand{\arraystretch}{\TableArrayStretch}
      \begin{tabularx}{\linewidth}{#2} % \linewidth로 꽉 채움
      \toprule
      #5 \\ % 헤더 매크로 확장
      \midrule
      #6   % 바디 매크로 확장
      \bottomrule
    \end{tabularx}
    }
  \end{table}
}

% --- 함수 3: 페이지가 나뉘는 긴 테이블 (longtable) ---
% 사용법: \newlongtable{<colspec>}{<caption>}{<label>}{<header-macro>}{<body-macro>}
% 예시:   \newlongtable{lcr}{캡션}{tab:my}{\myhead}{\mybody}
% ----------------------------------------------------------------------------
\newcommand{\newlongtable}[5]{%
  \centering % longtable은 table 환경 밖에서 centering
  {%
  \renewcommand{\arraystretch}{\TableArrayStretch}
  \begin{longtable}{#1}
    % --- 첫 페이지 헤더 ---
    \caption{#2} \label{#3} \\
    \toprule
    #4 \\ % 헤더 매크로 확장
    \midrule
    \endfirsthead
    
    % --- 다음 페이지 헤더 ---
    \caption[]{#2 (Continued)} \\ % 목차(LoT)에 중복 방지
    \toprule
    #4 \\ % 헤더 매크로 확장
    \midrule
    \endhead

    % --- 바닥글 ---
    \bottomrule
    \multicolumn{\NumCols}{r}{\textit{(Continued on next page)}} \\ % \NumCols는 이 테이블의 열 개수와 일치해야 함
    \endfoot
    
    % --- 마지막 바닥글 ---
    \bottomrule
    \endlastfoot

    % --- 본문 ---
    #5 % 바디 매크로 확장
  \end{longtable}
  }
  
}

% ---------------------------------------------------------------------------- %
%                         CUSTOM CODE BOX (listings)               %
% ---------------------------------------------------------------------------- %
