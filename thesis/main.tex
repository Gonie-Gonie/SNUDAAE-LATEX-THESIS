
% ---------------------------------------------------------------------------- %
%                        DOCUMENT DEFINITION AND COVERS                        %
% ---------------------------------------------------------------------------- %

\documentclass[11pt, oneside]{report}  % 기본 글꼴 (항상 첫 줄)   article로 바꿔야할듯???
% SNU Ph.D. Dissertation Template (English) — Updated to official 규격 (4×6배판)

% Source by 조형곤
% Revised by 박경수, 조형곤, 박철홍

% ---------------------------------------------------------------------------- %
%                                   PACKAGES                                   %
% ---------------------------------------------------------------------------- %


% ======= 페이지 레이아웃 =======
\usepackage[b5paper,
            top=2cm, bottom=1.5cm,
            left=3cm, right=3cm,
            headheight=15pt, headsep=15pt, footskip=15pt]{geometry}

% ======= 인코딩 및 글꼴 설정 =======
\usepackage[utf8]{inputenc}            % 입력 인코딩
\usepackage[T1]{fontenc}               % 출력 인코딩 (글꼴 인코딩)
\usepackage{kotex}                     % 한글 지원
\usepackage{newtxtext}                 % 영문 기본 글꼴 (Times New Roman (for English))

% ======= 참고문헌 =======
% \usepackage[round]{natbib}             % 참고문헌 인용 형식 (저자, 연도) (기존은 [숫자]임)
\usepackage[
  backend=biber,
  style=apa,   % apa   authoryear   ext-authoryear
  citestyle=apa,
  sorting=nyt,
  maxcitenames=1,   % 인용에서 보여줄 최대 저자 수
  mincitenames=1,   % 최소 두 명은 항상 이름 표시
  maxbibnames=99,   % 참고문헌 목록에 표시할 저자 수 (가능한 많이 유지)
  uniquename=false,
]{biblatex}

\addbibresource{reference.bib}

% 참고문헌 번호를 [1] 형식으로 보이게 조정
\DeclareFieldFormat{labelnumberwidth}{[#1]}
\renewcommand*{\labelnamepunct}{\addcolon\space}   % 강제로 번호 달기
\defbibenvironment{bibliography}
  {\begin{enumerate}
      \setlength{\labelwidth}{1.5em}
      \setlength{\leftmargin}{\labelwidth}
      \setlength{\labelsep}{0.5em}
      \setlength{\itemsep}{\bibitemsep}
      \setlength{\parsep}{\bibparsep}
   }
  {\end{enumerate}}
  {\item}


% ======= 시각 요소: 그림, 색상 =======
\usepackage{graphicx}                  % 그림 포함
\usepackage{xcolor}                    % 색상 지원
\usepackage[labelfont=bf, labelsep=period]{caption}   % : 대신 . 사용
\graphicspath{{./figures/}}            % 그림 경로 설정
\usepackage{subcaption}                % n by n 형식의 그림 배열
\usepackage{placeins}

% ======= 문단, 줄간격 등 레이아웃 설정 =======
\usepackage{setspace}                  % 줄간격 조정
\usepackage{indentfirst}               % 첫 문단 들여쓰기. 만약 개별문단별로 적용하려면 \indent 사용
\setstretch{1.7}                       % 기본 줄간격 1.7배 (170%)
\setlength{\parskip}{0.6em}            % 단락 사이 (세로) 여백
\setlength{\parindent}{2em}            % 단락 들여쓰기

% ======= 제목/번호/목차 관련 =======
\usepackage{titlesec}                  % 제목 포맷 커스텀
\usepackage{fancyhdr}                  % 헤더/푸터
\usepackage{tocloft}                   % 목차 스타일

% ======= 수식 및 수학 패키지 =======
% \usepackage{amsmath}                   % 수학 수식
\usepackage{amsmath, amssymb, amsfonts}
\usepackage{mathtools}  % for \coloneqq
% \numberwithin{equation}{chapter}       % 챕터별 수식 번호 (1.1)
% \renewcommand{\theequation}{Equation \thechapter.\arabic{equation}}   % (Equation 1.1)

% --- 표 관련 패키지 ---
\usepackage{booktabs}                  % \toprule 등 고급 표
\usepackage{longtable}                 % 페이지 넘기는 표
\usepackage{tabularx}                  % 가변 폭 표
\usepackage{multirow}                  % 셀 병합
\usepackage{makecell}                  % 셀 내 줄바꿈
\usepackage[tableposition=top]{caption}% 표 캡션 위 고정
\newcolumntype{C}{>{\centering\arraybackslash}X}

% ======= 참조 패키지 =======
\usepackage[hidelinks]{hyperref}       % 하이퍼링크
\usepackage[capitalise]{cleveref}      % cref 자동 참조
\usepackage{footmisc}                  % 각주 설정

% ======= 기타 디자인 =======
\usepackage{tcolorbox}                 % 박스 디자인

% ======= 코드류 =======
\usepackage{listings}                  % 코드삽입
\usepackage{lipsum}                    % 더미문장 \lipsum[1-2]


% ---------------------------------------------------------------------------- %
%                                    STYLE                                     %
% ---------------------------------------------------------------------------- %

% ======= Title formatting =======
\titleformat{\chapter}[hang]{\bfseries\LARGE}{Chapter\ \thechapter.}{1em}{}     % 장 제목
\titleformat{\section}[hang]{\bfseries\Large}{\thesection}{1em}{}               % 절 제목
\newcommand{\sectionbreak}{\clearpage}
\titleformat{\subsection}[hang]{\bfseries\large}{\thesubsection}{1em}{}    % 절의 하위제목
\titleformat{\subsubsection}[hang]{\bfseries\normalsize}{\thesubsection}{1em}{} % 하위제목의 하위제목
\titleformat{\paragraph}[hang]{\bfseries\normalsize}{\thesubsection}{1em}{}     % 소단락 제목
\titleformat{\subparagraph}[runin]{\bfseries\normalsize}{\thesubsection}{1em}{}  % 소단락의 하위제목

% ======= 서울대 맞춤 폰트 스타일 (필요시) =======
% \newcommand{\이름}[인자 수]{내용}
% \newcommand{\UnnumberedChapter}[1]{\fontsize{11pt}{18.7pt}\selectfont\textbf{#1}\\}   % {글꼴크기}{줄간격} 마지막 \\는 Enter 역할
\newcommand{\CustomToC}[1]{%
  \begingroup
    \fontsize{12pt}{20.4pt}\selectfont
    \setstretch{1.7}
    #1
  \endgroup
}   % 목차(Table of Contents) 스타일
\newlength{\basegap}        % 새 길이 변수를 선언 (표지에서 줄띄어쓰기용)
\setlength{\basegap}{14pt}  % 해당 변수에 14pt를 지정

% ======= Roman page numbers for frontmatter =======
\pagenumbering{roman}

% ======= Header/Footer =======
\pagestyle{fancy}
\fancyhf{}
\renewcommand{\headrulewidth}{0pt}
\fancyfoot[C]{\thepage}  % Footer centered


% ---------------------------------------------------------------------------- %
%                           TABLE AND FIGURE TEMPLATE                          %
% ---------------------------------------------------------------------------- %

\newcommand{\TableUppersideBlank}{10pt}
\newcommand{\TableLowersideBlank}{10pt}
\newcommand{\TableArrayStretch}{1.5}
\setlength{\belowcaptionskip}{\TableUppersideBlank}
\setlength{\textfloatsep}{\TableLowersideBlank}
\setlength{\LTpre}{\TableUppersideBlank}      % 테이블 위 (본문과의) 간격
\setlength{\LTpost}{\TableLowersideBlank}     % 테이블 아래 (본문과의) 간격

% --- 함수 1: 내용물 기준 테이블 (tabular) ---
% 사용법: \newcontenttable[<placement>]{<colspec>}{<caption>}{<label>}{<header-macro>}{<body-macro>}
% 예시:   \newcontenttable[ht!]{lcr}{캡션}{tab:my}{\myhead}{\mybody}
% ----------------------------------------------------------------------------
\newcommand{\newcontenttable}[6][htbp!]{%
  \begin{table}[#1]
    \centering
    \caption{#3}
    \label{#4}
    {%
    \renewcommand{\arraystretch}{\TableArrayStretch}
    \begin{tabular}{#2}
      \toprule
      #5 \\ % 헤더 매크로 확장
      \midrule
      #6   % 바디 매크로 확장 (마지막 \\는 바디 매크로 안에 포함되어야 함)
      \bottomrule
    \end{tabular}
    }
  \end{table}
}

% --- 함수 2: 좌우 꽉 찬 테이블 (tabularx) ---
% 사용법: \newfullwidthtable[<placement>]{<colspec>}{<caption>}{<label>}{<header-macro>}{<body-macro>}
% 예시:   \newfullwidthtable[ht!]{lX}{캡션}{tab:my}{\myhead}{\mybody}
% ----------------------------------------------------------------------------
\newcommand{\newfullwidthtable}[6][htbp!]{%
  \begin{table}[#1]
    \centering
    \caption{#3}
    \label{#4}
    {% 
      \renewcommand{\arraystretch}{\TableArrayStretch}
      \begin{tabularx}{\linewidth}{#2} % \linewidth로 꽉 채움
      \toprule
      #5 \\ % 헤더 매크로 확장
      \midrule
      #6   % 바디 매크로 확장
      \bottomrule
    \end{tabularx}
    }
  \end{table}
}

% --- 함수 3: 페이지가 나뉘는 긴 테이블 (longtable) ---
% 사용법: \newlongtable{<colspec>}{<caption>}{<label>}{<header-macro>}{<body-macro>}
% 예시:   \newlongtable{lcr}{캡션}{tab:my}{\myhead}{\mybody}
% ----------------------------------------------------------------------------
\newcommand{\newlongtable}[5]{%
  \centering % longtable은 table 환경 밖에서 centering
  {%
  \renewcommand{\arraystretch}{\TableArrayStretch}
  \begin{longtable}{#1}
    % --- 첫 페이지 헤더 ---
    \caption{#2} \label{#3} \\
    \toprule
    #4 \\ % 헤더 매크로 확장
    \midrule
    \endfirsthead
    
    % --- 다음 페이지 헤더 ---
    \caption[]{#2 (Continued)} \\ % 목차(LoT)에 중복 방지
    \toprule
    #4 \\ % 헤더 매크로 확장
    \midrule
    \endhead

    % --- 바닥글 ---
    \bottomrule
    \multicolumn{\NumCols}{r}{\textit{(Continued on next page)}} \\ % \NumCols는 이 테이블의 열 개수와 일치해야 함
    \endfoot
    
    % --- 마지막 바닥글 ---
    \bottomrule
    \endlastfoot

    % --- 본문 ---
    #5 % 바디 매크로 확장
  \end{longtable}
  }
  
}

% ---------------------------------------------------------------------------- %
%                         CUSTOM CODE BOX (listings)               %
% ---------------------------------------------------------------------------- %


% subjects
\newcommand{\EnglishTitle}{This is my THESIS}
\newcommand{\KoreanTitle}{내 학위논문 제목}
\newcommand{\EnglishKeywords}{Keyword, Keywordd, Keyworddd, Keywordddd, Keyworddddd}
\newcommand{\KoreanKeywords}{키워드, 키워드드, 키워드드드, 키워드드드드, 키워드드드드드}

% 심사
\newcommand{\PublishMonth}{February 2026}
\newcommand{\chairName    }{Gil-Dong Hong}
\newcommand{\viceChairName}{Gil-Dong ParK}
\newcommand{\examinerAName}{Gil-Dong Kim}
\newcommand{\examinerBName}{Gil-Dong Lee}
\newcommand{\examinerCName}{Kill-Dong Choi}

% personal info.
\newcommand{\MyEnglishName}{MyName KimLeeParkChoi}
\newcommand{\StudentNumber}{1234-56789}

% abstract
\newcommand{\EnglishAbstract}{
While dynamic simulation offers higher temporal fidelity and physical realism, its practical use remains limited due to the substantial input requirements, steep learning curve, and dependence on expert judgment. 
}
\newcommand{\KoreanAbstract}{
동적 시뮬레이션(dynamic simulation)은 시간적 해상도와 물리적 사실성을 높게 제공하지만, 방대한 입력 요구량, 높은 학습 곡선, 그리고 전문가의 판단에 대한 의존성으로 인해 실제 활용은 여전히 제한적이다.
}

% ======= 문서작성 시작 =======
\begin{document}

% 문서속성
\hypersetup{
  pdftitle={Ph.D Thesis},
  pdfauthor={Name},
  pdfkeywords={},
  bookmarksopen=true
}

% ======= Title Page (Cover, 외표지) =======
\begin{titlepage}
  \begin{center}
    {\fontsize{16pt}{27.2pt}\textbf{Ph.D. Dissertation of Engineering}}\\
      \vspace*{2cm}   % \vspace*{4em}   %\vspace*{4\basegap}
    {\fontsize{22pt}{37.4pt}\selectfont\textbf{\EnglishTitle}}\\   % Title of your Dissertation
      \vspace*{4\basegap}   %{\fontsize{16pt}{20pt}\selectfont - Subtitle of your Dissertation -}\\
    {\fontsize{16pt}{27.2pt}\selectfont\textbf{\KoreanTitle}}\\   % 국문 제목 작성
      %\vspace*{5\basegap}   
      \vfill
    {\fontsize{14pt}{23.8pt}\selectfont\textbf{\PublishMonth}}\\   % February or August
      \vspace*{4cm}   
    {\fontsize{16pt}{27.2pt}\selectfont\textbf{Graduate School of Engineering}}\\
    {\fontsize{16pt}{27.2pt}\selectfont\textbf{Seoul National University}}\\
    {\fontsize{14pt}{23.8pt}\selectfont\textbf{Architecture and Architectural Engineering Major}}\\   % Department Major
      \vspace*{1\basegap}   
    {\fontsize{16pt}{27.2pt}\selectfont\textbf{\MyEnglishName}}   % Full Name
      \vspace*{3cm}   

  \end{center}
\end{titlepage}

% ======= Title Page (Inner Title Cover, 속표지) =======
% 외표지와 동일하게 작성한다.
\begin{titlepage}
  \begin{center}
    {\fontsize{16pt}{27.2pt}\textbf{Ph.D. Dissertation of Engineering}}\\
      \vspace*{2cm}   % \vspace*{4em}   %\vspace*{4\basegap}
    {\fontsize{22pt}{37.4pt}\selectfont\textbf{\EnglishTitle}}\\   % Title of your Dissertation
      \vspace*{4\basegap}   %{\fontsize{16pt}{20pt}\selectfont - Subtitle of your Dissertation -}\\
    {\fontsize{16pt}{27.2pt}\selectfont\textbf{\KoreanTitle}}\\   % 국문 제목 작성
      %\vspace*{5\basegap}   
      \vfill
    {\fontsize{14pt}{23.8pt}\selectfont\textbf{\PublishMonth}}\\   % February or August
      \vspace*{4cm}   
    {\fontsize{16pt}{27.2pt}\selectfont\textbf{Graduate School of Engineering}}\\
    {\fontsize{16pt}{27.2pt}\selectfont\textbf{Seoul National University}}\\
    {\fontsize{14pt}{23.8pt}\selectfont\textbf{Architecture and Architectural Engineering Major}}\\   % Department Major
      \vspace*{1\basegap}   
    {\fontsize{16pt}{27.2pt}\selectfont\textbf{\MyEnglishName}}   % Full Name
      \vspace*{3cm}   

  \end{center}
\end{titlepage}

% ======= Approval Page (인준지) =======
\begin{titlepage}
  \begin{center}
    {\fontsize{22pt}{37.4pt}\selectfont\textbf{\EnglishTitle}}\\   % Title of your Dissertation
      \vspace*{1\basegap}
    {\fontsize{14pt}{23.8pt}\selectfont\textbf{Cheol-Soo Park}}\\   % Name of Examiner
      \vspace*{1\basegap}
    {\fontsize{16pt}{27.2pt}\selectfont\textbf{Submitting a Ph.D. Dissertation of}}\\
    {\fontsize{14pt}{23.8pt}\selectfont\textbf{Engineering}}\\   % Name of Examiner
      \vspace*{1\basegap}
    {\fontsize{14pt}{23.8pt}\selectfont\textbf{\PublishMonth}}\\   % February or August
      \vspace*{1\basegap}
    {\fontsize{16pt}{27.2pt}\selectfont\textbf{Graduate School of Engineering}}\\
    {\fontsize{16pt}{27.2pt}\selectfont\textbf{Seoul National University}}\\
    {\fontsize{14pt}{23.8pt}\selectfont\textbf{Architecture and Architectural Engineering Major}}\\
      \vspace*{1\basegap}
    {\fontsize{16pt}{27.2pt}\selectfont\textbf{\MyEnglishName}}   % Full Name
      \vspace*{1\basegap}\\
    {\fontsize{16pt}{27.2pt}\selectfont\textbf{Confirming the Ph.D. Dissertation written by}}\\
    {\fontsize{16pt}{27.2pt}\selectfont\textbf{\MyEnglishName}}   % Full Name
    \vfill
    {\fontsize{14pt}{23.8pt}\selectfont\textbf{\PublishMonth}}\\   % February or August
      \vspace*{1\basegap}
    \renewcommand{\arraystretch}{0.7}  % 행 간격 조절 (임의로 넣음 byKS)
    \newcommand{\signatureline}[1]{\makebox[6cm]{\centering\underline{\makebox[6cm]{\hfill #1 \hfill}}}}
    {
    \fontsize{14pt}{23.8pt}\selectfont
    \begin{tabular}{@{} l c c @{}} % 3열: 왼쪽, 가운데, 오른쪽 정렬
      Chair       & \signatureline{\chairName}      & (Seal) \\
      Vice Chair  & \signatureline{\viceChairName}  & (Seal) \\
      Examiner    & \signatureline{\examinerAName}  & (Seal) \\
      Examiner    & \signatureline{\examinerBName}  & (Seal) \\
      Examiner    & \signatureline{\examinerCName}  & (Seal) \\
    \end{tabular}
    \vspace*{2cm}  
  }
  \end{center}
\end{titlepage}


% ---------------------------------------------------------------------------- %
%                ABSTRACT, LIST OF CONTENTS, FIGURE, AND TABLES                %
% ---------------------------------------------------------------------------- %

% ======= Abstract =======
\newpage
\begin{center}
  {\fontsize{20pt}{34pt}\selectfont\textbf{Abstract}}
\end{center}
\addcontentsline{toc}{chapter}{Abstract}   % Content(목차)에 추가
\setlength{\parindent}{2em}  % 들여쓰기 2em
\setstretch{1.7}             % 줄간격 170%
\textbf{Keyword}: \EnglishKeywords\\
\textbf{Student Number}: \StudentNumber

\EnglishAbstract

% ======= Table of Contents =======
\newpage
\begin{center}
  {\fontsize{16pt}{27.2pt}\selectfont\bfseries Table of Contents}
\end{center}
    \vspace{-9em}  % 제목과 항목 사이 간격 조절 (임의조절)
\renewcommand{\contentsname}{}  % 자동 제목 비우기
\addcontentsline{toc}{chapter}{Table of Contents}   % Content(목차)에 추가
\begingroup
  \fontsize{12pt}{20.4pt}\selectfont  % 글자 크기 12pt, 줄간격 20.4pt (170%)
  \setstretch{1.7}                    % 줄간격 1.7로 고정
  \tableofcontents
\endgroup

% ======= List of Figures =======
\newpage
\begin{center}
  {\fontsize{16pt}{27.2pt}\selectfont\bfseries List of Figures}
\end{center}
    \vspace{-9em}  % 제목과 항목 사이 간격 조절 (임의조절)
\renewcommand{\listfigurename}{}  % 자동 제목 비우기
\addcontentsline{toc}{chapter}{List of Figures}   % Content(목차)에 추가
\begingroup
  \fontsize{12pt}{20.4pt}\selectfont  % 글자 크기 12pt, 줄간격 20.4pt (170%)
  \setstretch{1.7}                    % 줄간격 1.7로 고정
  \listoffigures
\endgroup

% ======= List of Tables =======
\newpage
\begin{center}
  {\fontsize{16pt}{27.2pt}\selectfont\bfseries List of Tables}
\end{center}
    \vspace{-9em}  % 제목과 항목 사이 간격 조절 (임의조절)
\renewcommand{\listtablename}{}  % 자동 제목 비우기
\addcontentsline{toc}{chapter}{List of Tables}   % Content(목차)에 추가
\begingroup
  \fontsize{12pt}{20.4pt}\selectfont  % 글자 크기 12pt, 줄간격 20.4pt (170%)
  \setstretch{1.7}                    % 줄간격 1.7로 고정
  \listoftables
\endgroup

% ======= Switch to Arabic numbering for main chapters =======
\cleardoublepage
\pagenumbering{arabic}

% ---------------------------------------------------------------------------- %
%                                  MAIN PAGES                                  %
% ---------------------------------------------------------------------------- %

\setlength{\emergencystretch}{3em} % 간격 여유있게 줄바꾸기
% ======= Main Chapters =======

% ---------------------------------------------------------------------------- %
%                                 INTRODUCTION                                 %
% ---------------------------------------------------------------------------- %

\chapter{Introduction}

\section{Research backgrounds}

\subsection{Purpose-driven evolution of building energy simulation tools}

Building energy simulation tools have evolved for decades, from manual calculation methods \parencite{cane1979modified,thom1966normal} to simulation \parencite{crawley2000energy,hittle1978building} and data-driven methods \parencite{rabl1988parameter,ruch1992four}, and recently to large language model based automated methods \parencite{jiang2024eplus,liu2025large}, along with the development of related fields such as gas laws, heat transfer, thermodynamics \parencite{dienel1997heat,mao2013literature,stamper1994ashraes}.

This divergence encapsulates the fundamental trade-off between fidelity and accessibility: dynamic tools continually extend the frontier of precision and control, while steady-state tools consolidate practicality, speed, and comparability for mainstream or regulatory applications (\cref{fig:dynamic-steady-comparison}).

\begin{figure}[htbp]
  \centering
  \includegraphics{dynamic_steadystate_comparison.png}
  \caption{Trajectories of dynamic and steady-state simulation tools}
  \label{fig:dynamic-steady-comparison}
\end{figure}

\newfullwidthtable{X X}
  {Table Example}         % caption
  {tbl:steady-vs-dynamic} % label
  { % header
  \multicolumn{1}{c}{\bfseries Steady-state simulator} & 
  \multicolumn{1}{c}{\bfseries Dynamic simulator}
  }
  { % body
  fast & slow and very slow and hard and slow and hard ans slow and muiltiline \\
  easy & hard \\
  }

\subsection{Emerging limits of the steady-state tools under contemporary building trends}

As buildings become lower-load and better insulated, integrate renewables and advanced HVAC (e.g., heat pumps with storage), and operate under occupant and policy driven controls, performance depends on temporal coupling among weather, occupancy, and control actions (start-up/shut-down and part-load behavior, short-term storage, and envelope thermal inertia) (\cref{fig:thermal-balance-comparison}).

\begin{figure}[htbp]
  \centering
  \subcaptionbox{Past high-energy buildings\label{fig:thermal-balance-past}}{%
    \includegraphics{VanDijk1-PastBuilding.png}}
  \hspace{1em}
  \subcaptionbox{Current low-energy buildings\label{fig:thermal-balance-current}}{%
    \includegraphics{VanDijk2-LowEnergyBuilding.png}}
  \caption{Comparison of thermal balances of past and current buildings \parencite{van2018epb}}
  \label{fig:thermal-balance-comparison}
\end{figure}

\subsection{Limitations in widespread adoption of dynamic simulators}

Accordingly, the principal impediments are delineated across three foundational input in the sections that follow: domains—geometry, HVAC systems, and internal loads.

\subparagraph{Geometry}

The result is predictable: rising effort for zone partitioning and boundary conditions, slower iteration, and weakened validation and reproducibility.

\subparagraph{HVAC Systems}

Consequently, QA/QC expands while iteration speed and reproducibility contract.

\subparagraph{Internal Loads}

This combination inflates modeling effort and injects uncertainty that is difficult to audit, eroding credibility and slowing adoption despite the central role of internal loads in performance outcomes.

\section{Thesis organization}

The subsequent chapters are organized as follows:

\begin{itemize}
  \item \textbf{Chapter 2} ...
  \item \textbf{Chapter 3} ...
  \item \textbf{Chapter 4} ...
  \item \textbf{Chapter 5} ...
  \item \textbf{Chapter 6} ...
\end{itemize}

% ---------------------------------------------------------------------------- %
%                               LITERATURE REVIEW                              %
% ---------------------------------------------------------------------------- %

\chapter{Previous efforts to ease use of dynamic simulation tools}


\section{Lightweighted engines}
hi

\section{User-friendly interfaces}

clicSAND for OSeMOSYS

\section{Building templates}

Templatized inputs (limited building shapes, ...) for specific building

Urban scale automation??

\section{Automated modelling approaches}

BIM2BEM, LLM

\section{Issues and limitations of the previous efforts}
hi

% ---------------------------------------------------------------------------- %
%                                    METHOD                                    %
% ---------------------------------------------------------------------------- %

\chapter{Normative simplified building components}

\section{Geometry}

\subsection{Definition of simplified geometry}
hi
\subsection{Simulation of the simplified geometry model in EnergyPlus}
hi
\subsection{Validation for the real-world buildings}
hi

\section{HVAC systems}

\subsection{Definition of simplified HVAC}
hi
\subsection{Definition of normative HVAC}
hi

\section{Internal loads}

\subsection{Definition of simplified profile}
hi
\subsection{Definition of normative profile}
hi


% ---------------------------------------------------------------------------- %
%                                IMPLEMENTATION                                %
% ---------------------------------------------------------------------------- %

\chapter{Python-based implementation}

\section{Data hierarchy and conversion procedures}

\subsection{Data hierarchy and classification}
hi
\subsection{Hierarchical conversions}
hi

\section{Building data structure for the simplified normative dynamic simulator}

\subsection{Tree-structured building data}
hi
\subsection{Type of the input variables}
hi
\subsection{Suggested data structure}
hi

\section{Python module structure}

\subsection{Functional module classification}
hi
\subsection{Api for user}
hi
\subsection{Auxiliary modules}
hi

\section{Algorithms for the data conversions}

\subsection{Building data conversion framework}
hi
\subsection{Simplified geometries to idf}
hi
\subsection{Simplified normative HVAC systems to idf}
hi
\subsection{Simplified or normative profiles to idf}
hi
% ---------------------------------------------------------------------------- %
%                                USER-INTERFACE                                %
% ---------------------------------------------------------------------------- %

\chapter{User-Interface}

\section{Spreadsheet based interface}

\subsection{Strength and limitations of the spreadsheet}
hi
\subsection{Spread-sheet based user input}
hi
\subsection{Dedicated launcher for spreadsheet}
hi

\section{Extensiblity for modern framework based interfaces}

\subsection{Strength of the modern frameworks}
hi
\subsection{Example of the React-framework based GUI}
hi

% ---------------------------------------------------------------------------- %
%                                  CONCLUSTION                                 %
% ---------------------------------------------------------------------------- %

\chapter{Conclusion}




% ---------------------------------------------------------------------------- %
%                             REFERENCE AND OTHERS                             %
% ---------------------------------------------------------------------------- %

% ======= Reference =======
\newpage
\titlespacing*{\chapter}{0pt}{-3em}{1em}   % left, before, after
\renewcommand{\bibname}{\normalfont\centering\fontsize{16pt}{27.2pt}\selectfont\textbf{References}}  % 'Bibliography'→'References로 변경
\addcontentsline{toc}{chapter}{Reference}  % 목차에 추가 
\begingroup
\setstretch{1.0} % ← 줄간격 조정 (기본보다 약간 좁게, 예: 1.0~1.1)
\printbibliography[title={References}] % biblatex : \parencite{}: (Author, Year), \textcite{}: Author (Year)
\endgroup
% ======= Korean Abstract =======
\newpage
\begin{center}
  {\fontsize{16pt}{27.2pt}\selectfont\textbf{국문초록}}
\end{center}
\addcontentsline{toc}{chapter}{국문초록}
\KoreanAbstract

\noindent\textbf{Keyword}: \KoreanKeywords\\
\textbf{Student Number}: \StudentNumber
% ======= The END =======
\end{document}